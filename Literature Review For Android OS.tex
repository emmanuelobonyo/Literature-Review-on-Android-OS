% Generated by GrindEQ Word-to-LaTeX 
\documentclass{article} %%% use \documentstyle for old LaTeX compilers

\usepackage[english]{babel} %%% 'french', 'german', 'spanish', 'danish', etc.
\usepackage{amssymb}
\usepackage{amsmath}
\usepackage{txfonts}
\usepackage{mathdots}
\usepackage[classicReIm]{kpfonts}
\usepackage[dvips]{graphicx} %%% use 'pdftex' instead of 'dvips' for PDF output

% You can include more LaTeX packages here 


\begin{document}

%\selectlanguage{english} %%% remove comment delimiter ('%') and select language if required


\noindent \textbf{Literature Review for Android OS}

\noindent ${}^{[1] }$Android is a mobile operating system developed by Google, based on a modified version of Linux Kernel and other open source software and designed primarily for touch screen devices. In addition, Google has also developed Android for cars, televisions and wrist watches, all with a specialized user interface associated with the device that it's running on to make the operation smoother. Android is written in Java -- for the User Interface -- and C and C++. 

\noindent Android, though, isn't the only mobile operating system on the market. iOS designed by Apple in California is Android's biggest competitor. Windows OS designed by Microsoft seems not to have taken off as much as the Personal Computer OS (Windows) has; this maybe is due to the fact that it came to the market late. That is not an excuse though since Android OS came into the market after iOS and managed to propel itself to the number mobile OS globally. A factor to Android's success is the fact it is open source and allows other companies like Samsung and LG to use it. iOS can only be used by iPhones (Phones designed by Apple). ${}^{[2] }$Google even allows you to skin it the way you want so that an Android Samsung device is different from a LG device, though they are running the same OS allows you to benefit from cross-platform functions. 

\noindent They're rumours though that Android copied iOS. Unlike Windows OS, Android OS greatly resembles iOS and in its early years, it was accused of being an iOS rip off. ${}^{[}$${}^{3]}$ Steve Jobs -- the former CEO of Apple -- actually vowed to do all he can to make sure that: `Android got destroyed', to use his own words. It not a secret that Android copied iOS; the evidence is there; from the features all the way to the GUI. But the great thing is that Android did not stop there. They made it better. 

\noindent 

\noindent \eject 
\[References:\] 
${}^{[1] }$wikipedia.org, `Android (operating system)', 2018. [Online]. Available: https://en.wikipedia.org/wiki/Android\_(operating\_system). [Accessed: 3 -- March -- 2018].

\noindent ${}^{[2]}$ trustedreviews.com, `Android vs iOS vs Windows 10 Mobile: Which mobile operating system is best?', 2018. [Online]. Available: http://www.trustedreviews.com/opinion/which-mobile-operating-system-is-best-2928049. [Accessed: 3 -- March -- 2018].

\noindent ${}^{[3] }$theguardian.com, `Did Android copy iOS? We asked Google's product manager...', 2018. [Online]. Available: https://www.theguardian.com/technology/2011/oct/25/google-android-copy-ios.  [Accessed: 3 -- March -- 2018].

\noindent \textbf{}



\end{document}

