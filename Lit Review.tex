% Generated by GrindEQ Word-to-LaTeX 
\documentclass{article} %%% use \documentstyle for old LaTeX compilers

\usepackage[english]{babel} %%% 'french', 'german', 'spanish', 'danish', etc.
\usepackage{amssymb}
\usepackage{amsmath}
\usepackage{txfonts}
\usepackage{mathdots}
\usepackage[classicReIm]{kpfonts}
\usepackage[dvips]{graphicx} %%% use 'pdftex' instead of 'dvips' for PDF output

% You can include more LaTeX packages here 


\begin{document}

%\selectlanguage{english} %%% remove comment delimiter ('%') and select language if required


\noindent 

\noindent 

\noindent 

\noindent 
\[1\] 


\noindent \textbf{1.0 Introduction}

\noindent Due to the large number of students at Makerere University, the student -- lecturer meaningful and resourceful collaboration has been greatly negatively affected. The recent exponential growth in the Information, Communication and Technology (ICT) sector, dedicates that this problem should not exist.

\noindent \textbf{2.0 Background to the Problem}

\noindent Makerere University has more than ${}^{[1] }$forty thousand students; all of which are pursuing different levels of education, from certificate level to ${}^{[2]}$ Doctor of Philosophy (PhD) Level. Compared with the much less\textit{ }number of teaching staff (lecturers) at the university -- one lecturer at times may be responsible for upto 500 students at a particular time) This a ratio of 1:400. This is absurd and obviously deems meaningful interaction between students and lecturers impossible. At times students have striked because apparently lecturers are teaching them. I believe if there was a medium that enabled safe and honest communication this problem would be almost nonexistent. 

\noindent Many times, my fclaasmates and I have gone to see certain lecturers only to find that they are very busy tending to other important matters or, even, unavailable in their offices or respective places of work for various reasons. This sometimes happens over and over again until one of the parties involved ends up giving up because they feel like valuable time is being wasted. This has meant that sometimes students have to rely on \textit{luck }to finally meet a lecturer -- that is if they don't meet during class time, which is not unusual seeing as due to the large number of students, some students easlily miss lecturers unnoticed, and sometimes the lecturers don't turn up for lectures (At times students have striked because apparently lecturers are teaching them. I believe if there was a medium that enabled safe and honest communication this problem would be almost nonexistent.). Often times communication is sent through third parties, for example class representatives -- and by the time the intended communication reaches the intended audience, it's either been slightly altered, miscommunicated or a tinge late. This, obviously, can create disastrous problems -- usually for the students -- depending on the urgency and complicacy of the information. 

\noindent \textbf{3.0 Problem Statement}

\noindent There is no meaningful, time efficient, and productive way for lecturers and students to interact in the university. Most of the interaction is usually by \textit{luck }-- which is not very time efficient or cost efficient, seeing as a student may go to a lecturer's office or place of work; only not to find them or, even, find then busy at that particular time; that is time and money wasted --\textit{ }or, even, in class -- which is obviously not sufficient and effective regarding the large number of students and also,obviously, does not facilitate later communication. Sometimes urgent communication and/or relevant information, for example class notes needs to passed on by the lecturer, though most times they are limited by an efficient cost effective method to pass on the information

\noindent \textbf{4.0  Objectives}

\noindent \begin{enumerate}
\item \textbf{1. }.\textbf{Main Objective}
\end{enumerate}

\noindent To find a way of creating meaningful and resourceful communication with the lecturers. 

\noindent \begin{enumerate}
\item \textbf{2. }.\textbf{Other Objective}
\end{enumerate}

\noindent To find the cause and develop a meaningful and resourceful form of interaction between the lecturers and the students. 

\noindent 

\noindent 


\end{document}

